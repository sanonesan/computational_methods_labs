
\documentclass[12pt, a4paper]{article}

\usepackage[utf8]{inputenc}
\usepackage[T1]{fontenc}
\usepackage[russian]{babel}
\usepackage{amsmath} 
\usepackage{amssymb} 
\usepackage{amsfonts} 
\usepackage[oglav,spisok,boldsect,eqwhole,figwhole,hyperref,hyperprint,remarks,greekit]{fn2kursstyle}

\usepackage{multirow}
\usepackage{supertabular}
\usepackage{multicol}
% Параметры титульного листа
\title{Исследование решений уравнения \\ Кортевега --- де Фриза в виде бегущей волны.}
\author{А.\,М.~Бобрун}
\supervisor{Г.\,В.~Гришина}
\group{ФН2-42Б}
\date{2022}

% Переопределение команды \vec, чтобы векторы печатались полужирным курсивом
\renewcommand{\vec}[1]{\text{\mathversion{bold}${#1}$}}%{\bi{#1}}
\newcommand\thh[1]{\text{\mathversion{bold}${#1}$}}
%Переопределение команды нумерации перечней: точки заменяются на скобки
\renewcommand{\labelenumi}{\theenumi)}
\begin{document}
	
	
	
	
	\newpage
	\begin{center}
		\Large{Лабораторная работа №1
			
			<<Прямые методы решения систем линейных \\ алгебраических уравнений>>}
	\end{center}
	\begin{flushleft}
		Студенты: Бобрун Александр, Егоров Александр
		
		Группа: ФН2-52Б
		
		Варианты: 4, 6
		
	\end{flushleft}
	\section-{Контрольные вопросы}
	
	\textbf{1. Каковы условия применимости метода Гаусса без выбора и с выбором ведущего элемента?}
	
	
	Метод Гаусса применим тогда и только тогда, когда все
	угловые миноры матрицы $A$ ненулевые. Если на главной диагонали возникают нулевые элементы, то Метод Гаусса без выбора ведущего элемента не применим. В этих случаях используется вариант алгоритма Гаусса с частичным либо полным выбором ведущего элемента. 
	
	\textbf{2.Докажите, что если $\det A \ne 0$, то при выборе главного элемента в столбце среди элементов, лежащих не выше главной диагонали, всегда найдется хотя бы один элемент, отличный от нуля. }
	
	Докажем это утверждение <<от противного>>.  Пусть $\det A \ne 0$. Так как элементарные преобразования строк и столбцов матрицы не изменяют ее определитель, рассмотрим шаг $k$ выбора главного элемента в столбце $k$. Пусть все элементы не выше главной диагонали нулевые:
	\[ A_k = 
	\begin{pmatrix}
		a_{11} & a_{12} & \ldots & a_{1(k-1)} & a_{1k} & a_{1(k+1)} & \ldots & a_{1n}\\
		0 & a_{22} & \ldots & a_{2(k-1)} & a_{2k} & a_{2(k+1)} & \ldots & a_{2n}\\
		\ldots & \ldots & \ldots & \ldots & \ldots & \ldots &   \ldots & \ldots \\
		0 & 0 & \ldots & a_{(k-1)(k-1)} & a_{(k-1)k} & a_{(k-1)(k+1)} & \ldots & a_{(k-1)n}\\
		0 & 0 & \ldots & 0 & 0 & a_{k(k+1)} & \ldots & a_{kn}\\
		0 & 0 & \ldots & 0 & 0 & a_{(k+1)(k+1)} & \ldots & a_{(k+1)n}\\
		\ldots & \ldots & \ldots & \ldots & \ldots & \ldots &   \ldots & \ldots \\
		0 & 0 & \ldots & 0 & 0 & a_{n(k+1)} & \ldots & a_{nn}\\
	\end{pmatrix}.
	\]
	Рассмотрим определитель матрицы $A_k$. Раскроем его по первому столбцу $k-1$ раз:
	$$\det A=\det A_k = a_{11} \cdot a_{22} \cdot \ldots \cdot a_{(k-1)(k-1)} \cdot 
	\begin{vmatrix}
		0 & a_{k(k+1)} & \ldots & a_{kn}\\
		0 & a_{(k+1)(k+1)} & \ldots & a_{(k+1)n}\\
		\ldots &   \ldots & \ldots &\ldots\\
		0 & a_{n(k+1)} & \ldots & a_{nn}\\
	\end{vmatrix} = 0.
	$$
	Но по предположению $\det A \ne 0$, что является противоречием, следовательно, всегда найдется хотя бы один элемент, отличный от нуля, при выборе главного элемента в столбце среди элементов, лежащих не выше главной диагонали, если $\det A \ne 0$.
	
	\textbf{3. В методе Гаусса с полным выбором ведущего элемента приходится не только переставлять уравнения, но и менять нумерацию неизвестных. Предложите алгоритм, позволяющий восстановить первоначальный порядок неизвестных.  }
	Для реализации предложенного нами алгоритма создается вектор $V$ длины $n$,  где на позиции $k$ стоит числo $k$, обозначающее порядковый  номер переменной в изначальной системе уравнений. Вектор $V$ имеет вид $V=(1,2,\dots ,k,\dots ,n)^T$.  Вместе с перестановкой столбцов $j_1$ и  $j_2$ матрицы $A$ меняем соотвествующие им элементы вектора $V$. После проделаных преобразований получаем вектор, в котором записана перестановка изначальных переменных. Запись $V[i]=m$ обозначает, что переменная, находящаяся изначально на позиции $m$, сейчас находится на позиции i. Для возобновления изначального порядка требуется переместить все столбцы в соотвествии с их позицией в векторе $V$.
	
	\textbf{4. Оцените количество арифметических операций, требуемых для $QR$-разложения произвольной матрицы $A$ размера $n \times n$.  }
	
	$QR$-разложения произвольной матрицы состоит из внешнего цикла от $i=0$ до $n-2$. В нем находится цикл от $j=i+1$ до $n$. Во втором цикле выполяются 5 операций для подсчета коэффицентов (2 деления, 1 извлечения корня, 2 возведения в степень) и цикл $k=0$ до $n-1$, в котором выполяются вычисления обоих матриц за 8 операций (произведения). Учитывая это, число операций будет равно величине 
	
	\[
	\sum\limits_{i=0}^{n-2} \left(\sum\limits_{j=i+1}^{n-1} \left(5+\sum\limits_{k=0}^{n}8\right)\right)=\sum\limits_{i=0}^{n-2} \left(\sum\limits_{j=i+1}^{n-1} \left(5+8\left(n+1\right)\right)\right)= \dfrac{1}{2}n(n-1)(8n+13).
	\]
	
	\textbf{5. Что такое число обусловленности и что оно характеризует? Имеется ли связь между обусловленностью и величиной определителя матрицы? Как влияет выбор нормы матрицы на оценку числа обусловленности? }
	
	Величину $\mathrm{cond} A = \| A^{-1} \| \cdot \| A \|$ называют числом обусловленности матрицы $A$. Число обусловленности характеризует чувствительность решения матрицы к малым погрешностям входных данных. 
	
	Явной зависимости между обусловленностью и величиной определителя матрицы нет. Например, рассмотрим две матрицы $A_1,A_2$ с одинаковыми определителями $\det A_1= \det A_2=1$
	\[ 
	A_1=
	\begin{pmatrix}
		1 & 0\\
		0 & 1
	\end{pmatrix},
	\quad
	A_2=
	\begin{pmatrix}
		2 & 3\\
		1 & 2
	\end{pmatrix}.
	\]
	Но их числа обусловленности различны
	\[ 
	\mathrm{cond} A_1=1,
	\quad
	\mathrm{cond} A_2=17.944.
	\]
	Рассмотрим другие две матрицы $A_3,A_4$ с одинаковыми числами обусловленности $\mathrm{cond} A_1=\mathrm{cond} A_2=1$
	\[ 
	A_3=
	\begin{pmatrix}
		2 & 0\\
		0 & 2
	\end{pmatrix},
	\quad
	A_3=
	\begin{pmatrix}
		3 & 0\\
		0 & 3
	\end{pmatrix}.
	\]
	Но их определители различны
	\[ 
	\det A_3=4,
	\quad
	\det A_4=9.
	\]
	
	Выбор нормы матрицы не влияет на оценку числа обусловленности, так как нормы эквивалентны.
	
	\textbf{Дополнение:} 
		\[ a)
	A=
	\begin{pmatrix}
		10000 & 0\\
		0 & 0.00001
	\end{pmatrix},
	\quad
		\mathrm{cond} A=10^{10},
	\quad
		\det{A}=1.
	\]
	
		\[ b)
	A=
	\begin{pmatrix}
		10000 & 0\\
		0 & 10000
	\end{pmatrix},
	\quad
	\mathrm{cond} A=1,
	\quad
	\det{A}=10^{10}.
	\]
	
	\textbf{6. Как упрощается оценка числа обусловленности, если матрица является: а) диагональной; б) симметричной; в) ортогональной; г) положительно определенной; д) треугольной}
	
	а) Так как у диагональной матрицы на диагонали стоят собственные значения, то число обусловленности будет меньше чем отношение $ \dfrac{ | \lambda_{max} |}{ | \lambda_{min} |}$, где $\lambda_{max}$ --- максимальный модуль элемента диагонали, $\lambda_{min}$ --- минимальный.
	
	б) Так как обратная к симметричной матрице --- тоже симметричная матрица, то можно вычислить только элементы, располагающиеся не выше (не ниже) главной диагонали. Это сократит количество операций в $\approx 2$ раза. Кроме того для симметричной матрицы кубическая и октаэдрическая нормы совпадают.
	
	в)Для ортогональной матрицы обратная матрица является транспонированной исходной.
	
	г) Если матрица положительно определена, то по критерию Сильвестра все главные миноры строго положительны. Следовательно, удобнее посчитать обратную матрицу методом Гаусса и провести оценку числа обусловленности, чем использовать метод Гаусса с выбором главного элемента. 
	
	д) Если матрица диагональная или треугольная, то ее определитель будет равен произведению диагональных элементов, а сами диагональные элементы будут являться собственными значениями.
	\looser{0.02}{ В итоге, число обусловленности будет меньше чем отношение $ \dfrac{ | \lambda_{max} |}{ | \lambda_{min} |}$, где $\lambda_{max}$ --- максимальный модуль элемента диагонали, $\lambda_{min}$ --- минимальный.}
	
	\textbf{7*. Применимо ли понятие числа обусловленности к вырожденным матрицам? }
	
	
	Не применимо. Так как у вырожденной матрицы бесконечное число решений. Следовательно, нельзя говорить об ее устойчивости к возмущению правой части.
	

\textbf{Дополнение:} Рассмотрим систему из двух уравнений с двумя неизвестными с вырожденной матрицей, то есть уравнения, задающие систему, линейно зависимы. В геометрической интерпритации уравнения представляют из себя параллельные прямые. Правая часть будет влиять только на расстояние между прямыми. Так как прямые не будут пересекаться, то решения системы, не будет существовать.

 
	\textbf{8*. \looser{0.02}{ В каких случаях целесообразно использовать метод Гаусса, а в каких — методы, основанные на факторизации матрицы? } }
	
	Когда матрица $A$ и столбец $b$ изменяются, целесообразнее использовать метод Гаусса. Так как в данном случае количество арифметических операций для произвольной матрицы размера $n \times n $ будет порядка $ \dfrac{n^3}{3} $, когда $QR$-разложение требует больших вычислительных затрат.
	{
		Когда же матрица $A$ остается неизменной, а столбец правой части меняется, целесообразнее использовать метод  $QR$-разложения. Вычислив разложение матрицы $A$ на ортогональную матрицу $Q$ и верхнетреугольную матрицу $R$, можно использовать только умножение вектора правой части на матрицу $Q^T$ и обратный ход метода Гаусса для системы $RX=Q^Tb$.}
	
\textbf{Дополнение:} Ортогональность матрицы $A$ можно проверять по ее свойствам: $AA^T=A^TA=E$, что эквивалентно $A^{-1}=A^T$.
	
	\textbf{ 9*. Как можно объединить в одну процедуру прямой и обратный ход метода Гаусса? В чем достоинства и недостатки такого подхода? }
	
	Сначала обнуляются элементы под главной диагональю столбца $i$, после чего обнуляются элементы над главной диагональю. Завершив операции, делим строчку на элемент главной диагонали, в итоге получая единичную матрицу и преобразованный вектор правой части, который является решением данной системы.
	К достоинствам можно отнести, что СЛАУ решается за одну процедуру. К недостаткам --- сложность алгоритма: в случае, если мы объединяем в одну процедуру прямой и обратный ход метода Гаусса, сложность алгоритма будет $ \dfrac{2 n^3}{3} $, в то время как, если мы не объединяем их, сложность алгоритма составит $ \dfrac{ n^3}{3} + \dfrac{n^2}{2} $.
	
	
	\textbf{10*. Объясните, почему, говоря о векторах, норму $ \| \cdot \| _1$ часто называют октаэдрической, норму $ \| \cdot \|_2$ — шаровой, а норму $ \| \cdot \|_{\infty}$ — кубической. }
	
	
	$ \| \cdot \| _1$ называют октаэдрической так как, в трехмерном пространстве единичный шар $K$, такой что $ \| K \| _1 \leq  1$, представляет собой октаэдр в трехмерном пространстве. 
	$ \| \cdot \| _2$ называют шаровой так как, в трехмерном пространстве единичный шар $K$, такой что $ \| K \| _2 \leq 1$, представляет собой шар в трехмерном пространстве. 
	$ \| \cdot \|_{\infty}$ называют кубической так как, в трехмерном пространстве единичный шар $K$, такой что $ \| K \| _{\infty} \leq 1$, представляет собой октаэдр в трехмерном пространстве. 
	
	\newpage
	
\end{document} 